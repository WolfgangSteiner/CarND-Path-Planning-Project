%%%%%%%%%%%%%%%%%%%%%%%%%%%%%%%%%%%%%%%%%
% Journal Article
% LaTeX Template
% Version 1.4 (15/5/16)
%
% This template has been downloaded from:
% http://www.LaTeXTemplates.com
%
% Original author:
% Frits Wenneker (http://www.howtotex.com) with extensive modifications by
% Vel (vel@LaTeXTemplates.com)
%
% License:
% CC BY-NC-SA 3.0 (http://creativecommons.org/licenses/by-nc-sa/3.0/)
%
%%%%%%%%%%%%%%%%%%%%%%%%%%%%%%%%%%%%%%%%%

%----------------------------------------------------------------------------------------
%	PACKAGES AND OTHER DOCUMENT CONFIGURATIONS
%----------------------------------------------------------------------------------------

\documentclass[twoside]{article}

%\usepackage[sc]{mathpazo} % Use the Palatino font
%\usepackage[UTF8]{fontenc} % Use 8-bit encoding that has 256 glyphs
\linespread{1.05} % Line spacing - Palatino needs more space between lines
\usepackage{microtype} % Slightly tweak font spacing for aesthetics
\usepackage[english]{babel} % Language hyphenation and typographical rules
\usepackage{amsmath}

\usepackage[hmarginratio=1:1,top=1in,left=1in,columnsep=20pt]{geometry} % Document margins
\usepackage[hang, small,labelfont=bf,up,textfont=it,up]{caption} % Custom captions under/above floats in tables or figures
\usepackage{booktabs} % Horizontal rules in tables

\usepackage{lettrine} % The lettrine is the first enlarged letter at the beginning of the text

\usepackage{enumitem} % Customized lists
\setlist[itemize]{noitemsep} % Make itemize lists more compact

\usepackage{abstract} % Allows abstract customization
\renewcommand{\abstractnamefont}{\normalfont\bfseries} % Set the "Abstract" text to bold
\renewcommand{\abstracttextfont}{\normalfont\small\itshape} % Set the abstract itself to small italic text

\usepackage{titlesec} % Allows customization of titles
\renewcommand\thesection{\Roman{section}} % Roman numerals for the sections
\renewcommand\thesubsection{\alph{subsection}} % roman numerals for subsections
\titleformat{\section}[block]{\large\scshape\centering}{\thesection.}{1em}{} % Change the look of the section titles
\titleformat{\subsection}{\itshape\bfseries}{\thesubsection.}{1ex}{} % Change the look of the section titles

% \usepackage{fancyhdr} % Headers and footers
% \pagestyle{fancy} % All pages have headers and footers
% \fancyhead{} % Blank out the default header
% \fancyfoot{} % Blank out the default footer
% %\fancyhead[C]{Running title $\bullet$ May 2016 $\bullet$ Vol. XXI, No. 1} % Custom header text
% \fancyfoot[RO,LE]{\thepage} % Custom footer text

\usepackage{titling} % Customizing the title section
\usepackage{hyperref} % For hyperlinks in the PDF
%\usepackage{flushend}
\usepackage{tabularx}
\usepackage{listings}
\usepackage{xcolor}
\usepackage{graphicx}
\usepackage{gensymb}
\usepackage{multirow}
\usepackage{stackengine}


\renewcommand{\c}{\text{c}}
\newcommand{\s}{\text{s}}
\newcommand{\pihalf}{\frac{\pi}{2}}
\newcommand{\T}[2]{\mbox{$_{#2}^{#1}{T}$}}
\newcommand{\R}[2]{\mbox{$_{#2}^{#1}{R}$}}
\newcommand{\code}[1]{{\texttt{#1}}}
\newcommand{\acos}{\text{acos}}
\newcommand{\asin}{\text{asin}}
\newcommand{\figref}[1]{Fig.~\ref{fig:#1}}
\newcommand{\tabref}[1]{Tab.~\ref{tab:#1}}

\def\stackalignment{l}
\newcommand{\imgframe}[2]{\topinset{\colorbox{black}{\textcolor{white}{#2}}}{\includegraphics[width=0.25\linewidth]{img/#1-#2.png}}{0cm}{0cm}}


\lstset{
  %frame=tb,
  language=Python,
  aboveskip=3mm,
  belowskip=3mm,
  showstringspaces=false,
  columns=flexible,
  basicstyle={\small\ttfamily},
  numbers=none,
  % numberstyle=\tiny\color{gray},
  % keywordstyle=\color{blue},
  % commentstyle=\color{dkgreen},
  % stringstyle=\color{mauve},
  breaklines=true,
  breakatwhitespace=true,
  tabsize=3
}
%----------------------------------------------------------------------------------------
%	TITLE SECTION
%----------------------------------------------------------------------------------------

\setlength{\droptitle}{-4\baselineskip} % Move the title up

\pretitle{\begin{center}\Huge\bfseries} % Article title formatting
\posttitle{\end{center}} % Article title closing formatting
\title{CarND Path Planning Project Report   } % Article title
\author{%
\textsc{Wolfgang Steiner} \\[0.5ex] % Your name
%\normalsize University of California \\ % Your institution
\normalsize \href{mailto:wolfgang.steiner@gmail.com}{wolfgang.steiner@gmail.com} % Your email address
%\and % Uncomment if 2 authors are required, duplicate these 4 lines if more
%\textsc{Jane Smith}\thanks{Corresponding author} \\[1ex] % Second author's name
%\normalsize University of Utah \\ % Second author's institution
%\normalsize \href{mailto:jane@smith.com}{jane@smith.com} % Second author's email address
}
\date{\today} % Leave empty to omit a date
\renewcommand{\maketitlehookd}{%
% \begin{abstract}
% \noindent
% \end{abstract}
}

%----------------------------------------------------------------------------------------

\begin{document}

% Print the title
\maketitle

%-------------------------------------------------------------------------------------------
%	ARTICLE CONTENTS
%-------------------------------------------------------------------------------------------
%%%%%%%%%%%%%%%%%%%%%%%%%%%%%%%%%%%%%%%%%%%%%%%%%%%%%%%%%%%%%%%%%%%%%%%%%%%%%%%%%%%%%%%%%%%%%%%%
\section{Introduction}
The goal of this project is to implement a path planning algorithm that enables a car to autonomously navigate highway traffic. The main requirements are keeping a safe distance to
the other cars, going as fast as the traffic and speed limits allow and to ensure a comfortable
ride for the passengers by limiting the maximum acceleration and jerk.

In this project I followed the approach discussed in \cite{werling2010optimal}: The vehicle
controller generates a multitude of candidate trajectories with different target speeds,
durations, and target driving lanes. For each candidate trajectory, a cost function is evaluated
that also involves the predicted trajectories of other nearby vehicles. The trajectory with
the smallest cost is then executed next. This mechanism is complemented by a simple state machine
that ensures the correct execution of the intended maneuvers.
%%%%%%%%%%%%%%%%%%%%%%%%%%%%%%%%%%%%%%%%%%%%%%%%%%%%%%%%%%%%%%%%%%%%%%%%%%%%%%%%%%%%%%%%%%%%%%%%
\section{Coordinate System and Waypoint Preprocessing}
For this project I decided to work exclusively in Frenet coordinates \cite{Frenet–S42:online}.
To facilitate this, I implemented the class \code{TWaypoints} that loads the waypoint coordinates from the supplied
\code{csv} file and fits two splines to the $x$- and $y$-coordinates, both having the longitudinal
$s$-coordinate as a parameter. In order to ensure a smooth transition at the starting point of the track,
the first and the last waypoint of the track are repeated while subtracting/adding the maximum
$s$-coordinate (i.e. length of the track).

When defining trajectories in Frenet coordinates $(s,d)$, transforming into cartesian coordinates
can be accomplished by evaluating the splines $f_x(s)$ and $f_y(s)$ at the s-coordinate and adding
the normal to the curve, multiplied by the d-coordinate:
\begin{equation}
  \begin{bmatrix}
    x \\ y
  \end{bmatrix}
  =
  \begin{bmatrix}
    f_x(s) \\
    f_y(s)
  \end{bmatrix}
  +
  d\cdot
  \begin{bmatrix}
    -\frac{d}{ds} f_y(s) \\
     \frac{d}{ds} f_x(s)
  \end{bmatrix}
  \label{eq:frenet_coordinates}
\end{equation}

The inverse transformation from Cartesian to Frenet coordinates is also required when
initializing the vehicle's position from the first simulator update and when updating the
positions of the other cars at each time step. The simulator does provide Frenet coordinates
but they turned out to be of insufficient accuracy. Given a position in Cartesian coordinates $(x,y)$, the corresponding $s$-coordinate is computed by a simple gradient descent solver \cite{Gradient43:online}
that minimizes the error term:
\begin{equation}
  J_{xy}(s) = \sqrt{(x - f_x(s))^2 + (y - f_y(s))^2}
\end{equation}
After having established the point on the spline that is closest to the target position,
the $d$-coordinate can be computed from \eqref{eq:frenet_coordinates}.
%%%%%%%%%%%%%%%%%%%%%%%%%%%%%%%%%%%%%%%%%%%%%%%%%%%%%%%%%%%%%%%%%%%%%%%%%%%%%%%%%%%%%%%%%%%%%%%%

%%%%%%%%%%%%%%%%%%%%%%%%%%%%%%%%%%%%%%%%%%%%%%%%%%%%%%%%%%%%%%%%%%%%%%%%%%%%%%%%%%%%%%%%%%%%%%%%
\section{Jerk Minimizing Trajectories}
%%%%%%%%%%%%%%%%%%%%%%%%%%%%%%%%%%%%%%%%%%%%%%%%%%%%%%%%%%%%%%%%%%%%%%%%%%%%%%%%%%%%%%%%%%%%%%%%
\subsection{Velocity Keeping Trajectories}
The most common state of the car is keeping a lane while maintaining a constant velocity, which
should be as close as possible to the speed limit, and keeping a safe distance to the leading
cars. As we are not interested in the exact position at the end of the trajectory, jerk minimizing
trajectories can be generated from \emph{quartic} polynomials \cite{werling2010optimal}:
\begin{align}
s(t) &= a_0 + a_1 t + a_2 t^2 + a_3 t^3 + a_4 t^4\\
\dot{s}(t) &= a_1 + 2 a_2 t + 3 a_3 t^2 + 4 a_4 t^3 \label{quartic2}\\
\ddot{s}(t) &= 2 a_2 + 6 a_3 t + 12 a_4 t^2 \label{quartic3}
\end{align}
The boundary conditions at the \emph{start} of the trajectory are the longitudinal position $s_0$,
velocity $\dot s_0$ and acceleration $\ddot s_0$ of the current state.
From this, the first three parameters of the polynomials can be directly determined:
\begin{align}
s(0) &= s_0 = a_0\\
\dot s(0) &= \dot s_0 = a_1\\
\ddot s(0) &= \ddot s_0 = 2 a_2
\end{align}
The boundary conditions at the \emph{end} of the trajectory are the longitudinal velocity $\dot s_1$
and acceleration $\ddot s_1$ (which is commonly zero). From \eqref{quartic2} and \eqref{quartic3}
we can now formulate the linear system of equations as:
\begin{align}
% s(T) &= s_1 = s_0 + \dot{s}_0 T + \frac{1}{2}\ddot{s}_0 T^2 + a_3 T^3 + a_4 T^4\\
\dot s(T) &= \dot s_1 = \dot s_0 + \ddot s_0 T + 3 a_3 T^2 + 4 a_4 T^3\\
\ddot s(T) &= \ddot s_1 = \ddot s_0 + 6 a_3 T + 12 a_4 T^2
\end{align}
This can be reformulated as the following matrix equation which in the code is solved for the
missing parameters $a_3$ and $a_4$ by applying the Householder transformation from the \emph{Eigen}
library:
\begin{equation}
  \begin{bmatrix}
    \dot s_1 - \dot s_0 - \ddot s_0 T \\
    \ddot s_1 - \ddot s_0
  \end{bmatrix}
=
\begin{bmatrix}
3 T^2 & 4 T^3 \\
6 T & 12 T^2
\end{bmatrix}
\cdot
\begin{bmatrix}
  a_3 \\ a_4
\end{bmatrix}
\label{quartic_matrix}
\end{equation}

\subsection{Lane Changing Trajectories}
For the longitudinal component of lane changing trajectories, the exact end position of in
the $s$-coordinate is still of no interest. For the lateral component $d$ this is different:
The trajectory should end at a specific $d$-coordinate, which is usually the center of the
target lane. So I combine a \emph{quartic} polynomial for the longitudinal
with a \emph{quintic} polynomial for the lateral component of the trajectory. The derivation
is similar to that of \eqref{quartic_matrix} and leads to a linear equation system of the form:
\begin{equation}
\small
\begin{bmatrix}
d_1 - d_0 - \dot d_0 T - \frac{\ddot d_0}{2}  T^2\\
      \dot d_1 - \dot d_0 - \ddot d_0 T\\
      \ddot d_1 - \ddot d_0
\end{bmatrix}
=
\begin{bmatrix}
  T^3&    T^4&    T^5\\
3 T^2&  4 T^3&  5 T^4\\
 6 T& 12 T^2& 20 T^3
\end{bmatrix}
\cdot
\begin{bmatrix}
a_3 \\ a_4 \\ a_5
\end{bmatrix}
\end{equation}


%%%%%%%%%%%%%%%%%%%%%%%%%%%%%%%%%%%%%%%%%%%%%%%%%%%%%%%%%%%%%%%%%%%%%%%%%%%%%%%%%%%%%%%%%%%%%%%%
\section{Tracking of Other Vehicles}
%%%%%%%%%%%%%%%%%%%%%%%%%%%%%%%%%%%%%%%%%%%%%%%%%%%%%%%%%%%%%%%%%%%%%%%%%%%%%%%%%%%%%%%%%%%%%%%%
Each time the vehicle controller is called by the simulator, sensor fusion data is received.
This data contains a unique ID for each vehicle along with position and speed in Cartesian
coordinates. The sensor data also contains the current position in Frenet coordinates, which,
unfortunately, is generally not accurate enough to be usable.\hspace{-.08em}\footnote{
The longitudinal Frenet coordinate $s$ is used as a starting point for
the gradient descent solver.}

The sensor data is passed to the \code{TSensorFusion} which maintains a hash table of
\code{TOtherCar} objects, indexed by ID. Each \code{TOtherCar} maintains its own Kalman Filter that is used to predict its future trajectory during the calculation of the safety distance cost. Each vehicle position is converted from Cartesian to Frenet coordinates $(s,d)$, while the total velocity is used for $\dot s$. With these state variables, the Kalman Filter of each other vechicle is updated at each iteration.

The implementation of the Kalman Filter has one peculiarity in that the current state can
be saved/restored before/after a prediction. This is required because the future vehicle positions have to be predicted far into the future (e.g. 10s), based on the most
recent measurement update (approx. every 20~-~60ms), in order to calculate safety distance
costs vs. candidate trajectories.
%%%%%%%%%%%%%%%%%%%%%%%%%%%%%%%%%%%%%%%%%%%%%%%%%%%%%%%%%%%%%%%%%%%%%%%%%%%%%%%%%%%%%%%%%%%%%%%%
\section{Trajectory Optimization}
%%%%%%%%%%%%%%%%%%%%%%%%%%%%%%%%%%%%%%%%%%%%%%%%%%%%%%%%%%%%%%%%%%%%%%%%%%%%%%%%%%%%%%%%%%%%%%%%
For finding optimal trajectories I follow the approach described in \cite{werling2010optimal}.
At every time step a multitude of trajectories are generated by varying the duration $T$ and
the target velocity $\dot s$. For lane changing trajectories, the target $d$-coordinate is also
(discretely) varied. These candidate trajectories are then evaluated using the following
cost function:
\begin{equation}
J_{traj} = k_v J_{v} + k_a J_{a} + k_j J_{j} + J_{\Delta d} + k_L J_{d} + k_{\rho} \underset{i}\max(J_{\rho}(i))
\end{equation}

Here, $J_{v}$, $J_{a}$ and $J_{j}$ are terms that penalize the deviation from the target velocity,
the maximum acceleration and the maximum jerk, respectively:

\begin{equation}
J_{v} = \frac{1}{T_H}\int\limits_0^{T_{H}} \left(\sqrt{\dot s(t)^2 + \dot d(t)^2} - v_{max}\right)^2 dt
\end{equation}

\begin{equation}
J_{a} = \frac{1}{T_H}\int\limits_0^{T_{H}}\ddot s(t)^2 + \ddot d(t)^2 dt
\end{equation}

\begin{equation}
J_{j} = \frac{1}{T_H}\int\limits_0^{T_{H}}\dddot s(t)^2 + \dddot d(t)^2 dt
\end{equation}


The safety distance of the trajectory in relation to the predicted trajectory of the $i$-th
other vehicle is calculated by the integral:

\begin{equation}
J_{\rho,i} = \frac{1}{T_H} \int\limits_0^{T_H} j_{\rho,i}(t) dt
\end{equation}

\begin{equation}
j_{\rho,i}(t) =
\begin{cases}
  0 & d > d_{min} \vee \rho(t) > \rho_s(v(t)) \\
  1 + \frac{1 - c_{min}}{1-e^{-\alpha \rho_s(v)}} \left(e^{-\alpha(\rho(t) - \rho_{0})} - 1 \right) & \text{otherwise}
  \end{cases}
\end{equation}
\begin{align}
\rho(t) &= s(t) - s_i(t) \\
\rho_s(v) &= v \cdot t_s + \rho_0
\end{align}



Here  is the velocity-dependent safety distance that is calculated based on
a constant-time rule with $t_s = 2s$. The parameter $\alpha$ can be used to tune the steepness
of the cost function,

When the lateral distance is greater than a safety threshold, the two cars drive on
different lanes and the cost is set to zero. Also, when the distance is greater than
the safety distance $d_s(v)$ the cost is also set to zero.

The middle lane is favorable to the other two lanes because it offers two possible direct
lane changes for overtaking cars. The cost $J_L$ therefore slightly penalizes driving
on the left and right lanes:
\begin{equation}
J_L =
\begin{cases}
0.0 & \text{current lane is center lane} \\
1.0 & \text{otherwise}
\end{cases}
\end{equation}

%%%%%%%%%%%%%%%%%%%%%%%%%%%%%%%%%%%%%%%%%%%%%%%%%%%%%%%%%%%%%%%%%%%%%%%%%%%%%%%%%%%%%%%%%%%%%%%%
\section{State Machine}
%%%%%%%%%%%%%%%%%%%%%%%%%%%%%%%%%%%%%%%%%%%%%%%%%%%%%%%%%%%%%%%%%%%%%%%%%%%%%%%%%%%%%%%%%%%%%%%%
I implemented an object-based state machine (\code{TStateMachine}) to handle the different
vehicle states (\code{TVehicleState}). For this it maintains a queue of currently active states
with the top of the queue being the current state. Every time the vehicle controller is called
by the simulator, the trajectory is first updated with the previous coordinates in order to
ensure its smoothness. Then the state machine executes the current vehicle state, which returns
an updated trajectory along with a pointer to the \emph{next} active state. This pointer can
have one of three different values:
\begin{enumerate}
\item When the current state returns a \code{this} pointer, it remains the current state.
\item Returning a \code{nullptr} signifies that the current state is finished. The previous
(or \emph{parent}) state will become current again.
\item If the current state returns a pointer to a newly constructed state, it thereby spawns a new
sub-task which will then be current.
\end{enumerate}


\subsection{Keep Lane State}
The \code{TKeepLaneState} is the default state of the car. It generates velocity keeping
trajectories that allow the car to keep the lane close to the speed limit, while also keeping
a safe distance to the leading cars. In addition, this state will generate candidate
trajectories that change to an adjacent lane and to the next-to-adjacent lane, if applicable.

If the car has to follow a slowly leading vehicle, the cost of this trajectory will
increase due to $J_{v}$ increasing. As soon as a trajectory with a different target lane
has a lower total cost, the lane keeping state will construct and return an object of type \code{TChangeLaneState}, which will then execute the lane change.


\subsection{Change Lane State}
The \code{TChangeLaneState} is responsible for executing a safe lane change maneuver. When this
state is executed for the first time after construction, the car has already started to move
towards the target lane. The lane changing state will generate target trajectories with varying
target velocities and durations in the $s$-direction. For the lateral direction, a quintic polynomial is generated with the target $d$-coordinate in the center of the target lane and with a fixed lane change duration.\hspace{-0.08em}\footnote{The lane change duration is fixed to one
value in order to limit the number of generated candidate trajectories.} For all of these candidate trajectories, the predicted trajectories of all nearby
cars in the traversed lanes are considered for evaluating the safety distance cost $J_\rho$.

In addition, fall-back trajectories are generated that return the car to the original lane.
These trajectories enable the car to react to other vehicles' sudden
lane changes while executing the maneuver. The safety distance cost of the best lane
change trajectory is compared to that of the best fall-back trajectory. Should the safety
distance cost of the lane change increase, the fall-back trajectory will be selected and the
safer lane will be set as the new target lane, thus avoid collisions with other cars mid-maneuver.

As soon as the lateral Frenet coordinate $d$ is close to its target value, the change lane state
will return the \code{nullptr}. Consequently the lane change maneuver will end, the object is
deallocated, and trajectory planning is returned to the parent keep lane state object.

%%%%%%%%%%%%%%%%%%%%%%%%%%%%%%%%%%%%%%%%%%%%%%%%%%%%%%%%%%%%%%%%%%%%%%%%%%%%%%%%%%%%%%%%%%%%%%%%
\section{Results}
%%%%%%%%%%%%%%%%%%%%%%%%%%%%%%%%%%%%%%%%%%%%%%%%%%%%%%%%%%%%%%%%%%%%%%%%%%%%%%%%%%%%%%%%%%%%%%%%
The implemented path planner is able to safely navigate through highway traffic near the speed
limit. Owing to the large planning horizon of 10s, the car slows down or switches lane with
foresight when approaching a slower leading car from behind. It also keeps clear of faster
cars that come up from behind. An example of a lane changing maneuver is shown in \figref{fallback}. In the maneuver shown, the lane change is interrupted by another car that
overtakes on the middle lane. Because the lane changing trajectory has become unsafe, the
controller steers the car back to the original lane and lets the overtaking car pass before
initiating and completing the lane change in a second attempt.

\begin{figure}[ht]
\centering
\imgframe{fallback}{1}\imgframe{fallback}{2}\imgframe{fallback}{3}\imgframe{fallback}{4}\vspace{-1mm}
\imgframe{fallback}{5}\imgframe{fallback}{6}\imgframe{fallback}{7}\imgframe{fallback}{8}\vspace{-1mm}
\imgframe{fallback}{9}\imgframe{fallback}{10}\imgframe{fallback}{11}\imgframe{fallback}{12}
\caption{\label{fig:fallback}Example of a fall-back maneuver during lane changing. The car initiates a lane
change maneuver (1-2) but then resets the target lane to the original lane
(3). It then safely returns to the original lane (4-5). The
car that caused the fall-back maneuver is seen overtaking on the middle lane in (6-7).
After the red car has passed, a new lane change maneuver is initiated and successfully
completed (8-12).}
\end{figure}


%%%%%%%%%%%%%%%%%%%%%%%%%%%%%%%%%%%%%%%%%%%%%%%%%%%%%%%%%%%%%%%%%%%%%%%%%%%%%%%%%%%%%%%%%%%%%%%%
\section{Outlook}
%%%%%%%%%%%%%%%%%%%%%%%%%%%%%%%%%%%%%%%%%%%%%%%%%%%%%%%%%%%%%%%%%%%%%%%%%%%%%%%%%%%%%%%%%%%%%%%%
There are many ways in which the path planner could be improved. More states could be added
to the state machine that would implement following of leading vehicles at a predetermined
safety distance or merging into an adjacent lane between other cars. Sometimes the car
gets stuck behind a slow car with a second car slightly ahead on the middle lane. If the
planner would slow down for a moment, it could direct the car to the third lane, overtaking
both other cars. But a behavior like this would require to combine several trajectories 

%----------------------------------------------------------------------------------------
%	REFERENCE LIST
%----------------------------------------------------------------------------------------
\bibliography{main}
\bibliographystyle{ieeetr}
%----------------------------------------------------------------------------------------

\end{document}
